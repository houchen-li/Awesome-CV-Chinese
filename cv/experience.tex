%-------------------------------------------------------------------------------
%	SECTION TITLE
%-------------------------------------------------------------------------------
\cvsection{Experience}


%-------------------------------------------------------------------------------
%	CONTENT
%-------------------------------------------------------------------------------
\begin{cventries}

%---------------------------------------------------------
  \cventry
    {本科生} % Job title
    {清华大学求学期间} % Organization
    {中国,北京} % Location
    {2012年9月 - 2016年6月} % Date(s)
    {
      \begin{cvitems} % Description(s) of tasks/responsibilities
        \item {在理论力学、量子力学、电动力学、统计力学、数值分析、计算物理、天体物理、广义相对论、非线性动力学与混沌等课程的学习中表现优异}
        \item {对数值算法产生兴趣并尝试编写\textit{Numerical Recipes}中的例题}
        \item {开始在Linux上使用C、C++编程}
        \item {开始使用Python学习数据可视化}
      \end{cvitems}
    }

%---------------------------------------------------------
  \cventry
    {访问学生} % Job title
    {暑期研修:平面库埃特流中边界态的统计分布(Prof. Bruno Eckhardt)} % Organization
    {德国,马尔堡} % Location
    {2015年6月 - 2015年8月} % Date(s)
    {
      \begin{cvitems} % Description(s) of tasks/responsibilities
        \item {在集群上运行平面库埃特流模拟计算,总计算次数超过5000次}
        \item {确定微扰条件下层流向湍流转变的随机概率分布}
        \item {学习流场可视化}
        \item {积累组织管理大型C++项目的经验}
      \end{cvitems}
    }

%---------------------------------------------------------
  \cventry
    {本科生} % Job title
    {学士学位论文:高维混沌系统中临界周期轨道的变分解法(兰岳恒教授)} % Organization
    {中国,北京} % Location
    {2016年2月 - 2016年7月} % Date(s)
    {
      \begin{cvitems} % Description(s) of tasks/responsibilities
        \item {编写一个超过5000行的C++数值模拟程序}
        \item {使用高级数值技巧来进行大矩阵操作、大矩阵求逆、最小值搜索、参数优化}
        \item {成功找出高维混沌系统中的周期轨道}
        \item {验证变分法的可靠性}
        \item {改进原有算法中矩阵求逆效率}
      \end{cvitems}
    }

%---------------------------------------------------------
  \cventry
    {硕士生} % Job title
    {马尔堡-菲利普大学求学期间} % Organization
    {德国,马尔堡} % Location
    {2017年4月 - 2020年6月} % Date(s)
    {
      \begin{cvitems} % Description(s) of tasks/responsibilities
        \item {完成高等量子力学、高等计算物理、量子场论、随机过程等课程的学习}
        \item {实践诸如伊辛模型、蒙特卡洛算法、随机游走的模拟}
        \item {学习并掌握许多开源数值算法库的使用,诸如Eigen3、Boost、openMPI、HDF5、fftw、numpy、scipy、matplotlib}
      \end{cvitems}
    }

%---------------------------------------------------------
  \cventry
    {硕士生} % Job title
    {研究课题:新型边界条件瑞利--伯纳德对流派生的洛伦兹模型(Prof. Bruno Eckhardt)} % Organization
    {德国,马尔堡} % Location
    {2017年6月 - 2018年9月} % Date(s)
    {
      \begin{cvitems} % Description(s) of tasks/responsibilities
        \item {使用新型边界条件的微分方程来构造洛伦兹模型}
        \item {计算新边界条件下洛伦兹模型的稳定性}
        \item {增进对于微分方程和线性代数的理解}
      \end{cvitems}
    }

%---------------------------------------------------------
  \cventry
    {硕士生} % Job title
    {研究课题:准线性近似下瑞利--伯纳德对流的数值解法(Prof. Bruno Eckhardt)} % Organization
    {德国,马尔堡} % Location
    {2018年10月 - 2019年7月} % Date(s)
    {
      \begin{cvitems} % Description(s) of tasks/responsibilities
        \item {使用平均场近似简化一个热对流系统(瑞利--伯纳德对流)}
        \item {编写用于演算此简化流体系统的C++程序}
        \item {将运算得到的流体数据进行可视化}
        \item {检测到在频空间中一种新的能量分布}
      \end{cvitems}
    }

%---------------------------------------------------------
  \cventry
    {硕士生} % Job title
    {硕士学位论文:压强驱动的反馈控制策略在泊肃叶流中的应用(Dr. Moritz Linkmann)} % Organization
    {德国,马尔堡} % Location
    {2019年9月 - 2020年4月} % Date(s)
    {
      \begin{cvitems} % Description(s) of tasks/responsibilities
        \item {使用反馈控制策略维持稳定超高雷诺数下的层流结构}
        \item {在集群上进行了高达30000 core hours的模拟}
        \item {研究反馈控制策略下守恒律的统计表征}
      \end{cvitems}
    }

%---------------------------------------------------------
  \cventry
    {研究助理} % Job title
    {研究课题:检测拓扑超导体的大数据策略(Prof. Jeroen van den Brink)} % Organization
    {德国,德累斯顿} % Location
    {2020年7月 - 2021年4月} % Date(s)
    {
      \begin{cvitems} % Description(s) of tasks/responsibilities
        \item {使用密度泛函理论计算材料中的电子运动}
        \item {从互联网上选取了6000多种材料结构}
        \item {设计一套自动提交模拟计算的自动化流程系统}
        \item {从TB量级的数据中提取关键信息}
        \item {把最终结果压缩保存为HDF5格式}
        \item {成功区分出导体和绝缘体材料}
      \end{cvitems}
    }
    
  \cventry
    {研究助理} % Job title
    {研究课题:手性晶体材料中的轨道角动量极化现象(Prof. Claudia Felser)} % Organization
    {德国,德累斯顿} % Location
    {2021年4月 - 2022年1月} % Date(s)
    {
      \begin{cvitems}
        \item {使用密度泛函理论计算布里渊区中各格点上的轨道角动量分布}
        \item {开发出一个用于执行矩阵操作和数据图像化的Python包}
        \item {成功找出PdGa晶体中的轨道角动量极化现象}
      \end{cvitems}
    }
    

%---------------------------------------------------------
\end{cventries}
